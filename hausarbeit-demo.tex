\documentclass{hausarbeit-jura}
\usepackage{blindtext}

%% Optionen, falls man gewisse Standards aus der Klasse ändern möchte
\usepackage{moreenum} % für griechische Zählung
\usepackage{titlesec} % zum Anpassen der Überschriftengröße/-dicke

%% footnotes nun unter dem bottom float (sonst können sie manchmal zu weit oben sein)
%%\usepackage[bottom]{footmisc}


%% only use with pdftex and luatex
\usepackage[final,tracking=smallcaps,expansion=alltext,protrusion=true]{microtype}%
\SetTracking{encoding=*,shape=sc}{50}%



%% sets font for xetex and lualatex, only use with these engines
\setmainfont{Arial} % normalerweise Times New Roman
\renewcommand{\footnotesize}{\fontsize{10pt}{12pt}\selectfont} % normalerweise ist es 9pt; der erste Wert ist die Größe, der zweite der Abstand
%% \setlength{\footskip}{15pt}
%%\setlength{\footnotesep}{12pt}
%%\setlength{\skip\footins}{2pt}
%%\hangfootparskip


\usepackage{ragged2e}
\usepackage[ragged,hang,bottom,multiple]{footmisc}

%\usepackage{dblfnote}

\usepackage{etoolbox}
\makeatletter
\patchcmd\@makefntext
  {\@makefnmark\hss}
  {\hss\@makefnmark\space}
  {}{}
\makeatother

%\setlength{\footnotesep}{0.1cm}


%% Abstände und Schriftgröße und -dicke für Überschriften
\titleformat{\chapter}
  {\normalfont\fontsize{12}{12}\bfseries}{\thechapter}{1em}{}
  % \titlespacing*{\chapter}{0pt}{1em}{0.5em} % reduziert Abstand nach Kapiteln
\titleformat{\section}
  {\normalfont\fontsize{12}{12}\bfseries}{\thesection}{1em}{}
\titleformat{\subsection}
  {\normalfont\fontsize{12}{12}\bfseries}{\thesubsection}{1em}{}
\titleformat{\subsubsection}
  {\normalfont\fontsize{12}{12}\bfseries}{\thesubsubsection}{1em}{}
\titleformat{\subthreesection}
  {\normalfont\subthreefontsize{12}{12}\bfseries}{\thesubthreesection}{1em}{}
\titleformat{\subfoursection}
  {\normalfont\fontsize{12}{12}\bfseries}{\thesubfoursection}{1em}{}
\titleformat{\subfivesection}
  {\normalfont\fontsize{12}{12}\bfseries}{\thesubfivesection}{1em}{}
\titleformat{\subsixsection}
  {\normalfont\fontsize{12}{12}\bfseries}{\thesubsixsection}{1em}{}
\titleformat{\subsevensection}
  {\normalfont\fontsize{12}{12}\bfseries}{\thesubsevensection}{1em}{}


%% setzt die Zähler für Überschriften (alle nötig, sofern Makro gesetzt wird, und nicht schon in Klasse geändert)
%% drei geändert, alle eines nach unten verschoben
\renewcommand{\thechapter}{\Alph{chapter})} % geändert
\renewcommand{\thesection}{\Roman{section})} % geändert
\renewcommand{\thesubsection}{\arabic{subsection})}
\renewcommand{\thesubsubsection}{\alph{subsubsection})}
\renewcommand{\thesubthreesection}{\alph{subthreesection}\alph{subthreesection})}
\renewcommand{\thesubfoursection}{(\arabic{subfoursection})}
\renewcommand{\thesubfivesection}{(\alph{subfivesection})}
\renewcommand{\thesubsixsection}{(\alph{subsixsection}\alph{subsixsection})}
\renewcommand{\thesubsevensection}{\boldmath\greek{subsevensection})} %geändert

%% ändert die Einrückung für section, subsection und subsevensection, soll hier als Beispiel dienen, wie man Überschriftenzähler anpa
\makeatletter
\renewcommand{\jubo@settocindents}{%
  \if@chapterprefix%
  \settowidth{\chapternumwidth}{\chaptername~999\chapterextra\space}% warum drei x 9? zweimal ist zu knapp
  \else% chaptersuffix:
  \settowidth{\chapternumwidth}{999.~\chaptername\chapterextra\space}%
  \fi%
  \settowidth{\sectionnumwidth}{III)\ } % geändert
  \settowidth{\subsectionnumwidth}{8)\ } % geändert
  \settowidth{\subsubsectionnumwidth}{b)\ }%
  \settowidth{\subthreesectionnumwidth}{bb)\ }%
  \settowidth{\subfoursectionnumwidth}{(8)\ }%
  \settowidth{\subfivesectionnumwidth}{(b)\ }%
  \settowidth{\subsixsectionnumwidth}{(bb)\ }%
  \settowidth{\subsevensectionnumwidth}{$\alpha$)\ } % geändert
  \settowidth{\subeightsectionnumwidth}{(iii)\ }%
  \if@chapterprefix%
  \setlength{\sectiontocindent}{\chapternumwidth}%
  \else%
  \settowidth{\@tempdima}{\bfseries 99.\space}%
  \setlength{\sectiontocindent}{\@tempdima}%
  \fi
  \if@fixtocindent%
  \setlength{\sectiontocindent}{\jb@defaulttocindent@length}% = 1.5em per default
  \fi%
  \setlength{\subsectiontocindent}{\sectiontocindent}%
  \addtolength{\subsectiontocindent}{\sectionnumwidth}%
  \setlength{\subsubsectiontocindent}{\subsectiontocindent}%
  \addtolength{\subsubsectiontocindent}{\subsectionnumwidth}%
  \setlength{\subthreesectiontocindent}{\subsubsectiontocindent}%
  \addtolength{\subthreesectiontocindent}{\subsubsectionnumwidth}%
  \setlength{\subfoursectiontocindent}{\subthreesectiontocindent}%
  \addtolength{\subfoursectiontocindent}{\subthreesectionnumwidth}%
  \setlength{\subfivesectiontocindent}{\subfoursectiontocindent}%
  \addtolength{\subfivesectiontocindent}{\subfoursectionnumwidth}%
  \setlength{\subsixsectiontocindent}{\subfivesectiontocindent}%
  \addtolength{\subsixsectiontocindent}{\subfivesectionnumwidth}%
  \setlength{\subsevensectiontocindent}{\subsixsectiontocindent}%
  \addtolength{\subsevensectiontocindent}{\subsixsectionnumwidth}%
  \setlength{\subeightsectiontocindent}{\subsevensectiontocindent}%
  \addtolength{\subeightsectiontocindent}{\subsevensectionnumwidth}%
  \setlength{\paragraphtocindent}{\subeightsectiontocindent}%
  \addtolength{\paragraphtocindent}{\subeightsectionnumwidth}%
  \setlength{\subparagraphtocindent}{\paragraphtocindent}%
  \addtolength{\subparagraphtocindent}{\jb@defaulttocindent@length}%
}
\makeatother

%% das setzt einen variablen Abstand zwischen Absätzen
%% \setlength{\parskip}{1ex plus 0.5ex minus 0.2ex}
%% sollte nur genutzt werden, wenn folgende Funktion aktiviert ist:
%% deaktiviere indent überall
%% \setlength\parindent{0pt}

%% legt den Rand für den Hauptteil fest
\setpgmain{left=2cm,right=5cm,top=2cm,bottom=2cm,footskip=0.5cm,marginparwidth=5cm,verbose}

%% legt den Rand für den Vorspann fest;
\setpgfront{left=3.5cm,right=2cm,top=2cm,bottom=2cm,footskip=0.5cm,marginparwidth=5cm,verbose}

%% ermöglicht Zeilenabstand von Word (nur im Hauptteil, nicht in den Fußnoten)
%% https://tex.stackexchange.com/a/355610
\makeatletter
\renewcommand{\onehalfspacing}{%
  \normalsize
  \ifthenelse{\@ptsize = 0}% 10pt
    {\renewcommand{\baselinestretch}{1.25}}%
    {}
  \ifthenelse{\@ptsize = 1}% 11pt
    {\renewcommand{\baselinestretch}{1.21}}%
    {}
  \ifthenelse{\@ptsize = 2}% 12pt
    {\renewcommand{\baselinestretch}{1.5}}% normalerweise 1.24, das gibt jetzt den verkappten Zeilenabstand von Word.
    {}
  \normalsize
}
\makeatother

%% für Verlinkung in PDF
\usepackage[hidelinks]{hyperref}

%% setzt biblatex als bib-Klasse mit biblatex-jura2 als Style
\usepackage[%
  style     = jura2,%
  backend   = biber,%
  sorting   = nty,%
  sortcites = true,%
  maxnames  = 4,%
  minnames  = 4,%
  articlein = false,%
  %innamebeforetitle = true,
  date      = comp,%short
  urldate   = comp,
  dateabbrev = true,
  useprefix = true,%
  isbn      = false,%
  doi       = true,%
  backref   = false,%
  abbreviate = true,%
]{biblatex}
\addbibresource{hausarbeit-demo.bib}

\begin{document}

%% optional
\frontmatter

%%% Vorspann mit Titel, Sachverhalt und den Verzeichnissen
\title{Hausarbeit}
\subtitle{Übung für Anfänger im öffentlichen Recht}
%% \author{Otto Normalverbraucher\\Musterweg 12\\12345 Musterstadt\\123456\\3. Fachsemester}
\author{Otto Normalverbraucher\\Musterweg 12\\12345 Musterstadt}
\matrikelnummer{123456}
\semester[3. Fachsemester]{Wintersemester 2015/2016}
\prof{Prof.\,Dr. X. Y.}
\date{15. Februar 2016}
\maketitle

\tableofcontents %Inhaltsverzeichnis

%% Bibliographie auf neuer Seite, leicht abgeändert übernommen aus biblatex-jura2
%% kann hier gesetzt werden, oder am Ende
%% \newpage
%% \sloppy
%% \addcontentsline{toc}{chapter}{Verzeichnis der exemplarisch verwendeten Literatur} % fügt Lit-verz. ins Inhaltsverzeichnis ein
%% \printbibliography[nottype=jurisdiction]
%% \pagenumbering{Roman} % This command sets the page numbers to uppercase Roman numerals.
%% \setcounter{page}{3} % muss manuell gesetzt werden
%%%%%%%%%%%%%%%%%%%%%%%%%%%%%%%%%%%%%%%%%%%%%%%%%%%%%%%%%%%%%%%%%%%%%%%%%%%%%

%% optional
%% \mainmatter

%% ab hier Hauptteil
\chapter{Ein erstes Kapitel}\label{chap:ErstesKapitel}
%% \TODO{Da fehlt noch was!}

\section{Abschnitt}
\blindtext[4]
\subsection{Ein Unterabschnitt}
%% Von dem Stylepaket biblatex-jura2 übernommen, für Erklärungen in dessen Dokumentation schauen.
%% https://ctan.org/pkg/biblatex-jura2

Erste Fußnote.\footcite[33]{larenz:methoden}

Zweite Fußnote. \footcite[§~7 Rn.~7]{maurer:allgverwr}

Dritte Fußnote. \footcite[Rn.~204]{medicus:br}

Vierte Fußnote. \footcite{stamm:verzinsung}

Fünfte Fußnote. \footcite[3]{stamm:zweiter}

Sechste Fußnote. \footcite[vgl.][517]{gehrlein:vollstreckung}

Siebte Fußnote. \footcite[(Ellenberger)§~119 Rn.~4]{palandt}

Achte Fußnote. \footnote{\cite[(Flechtner)§~2059 BGB Rn.~1]{erbr}; siehe auch \cite[(Kindler)Vorbem. §§~1--7 (Kaufmannsbegriff) Rn.~3]{ebjs1}.\baselineskip12pt}

Neunte Fußnote. \footnote{vgl. \cite[Schmidt-Aßmann][Art.~19 Abs.~4 Rn.~36]{maunzduerig}; a.\,A. \cite[Diemer][§~151 Rn.~2]{kkstpo}.\baselineskip12pt}

Zehnte Fußnote. \footcite[Matusche-Beckmann][§~5 Rn.~14]{hdbversr}

Elfte Fußnote. \footcite[(Matusche-Beckmann)§~5 Rn.~14 (falsch zitiert)]{hdbversr}

Zwölfte Fußnote. \footnote{\cite[(Bearbeiter)§~433 Rn.~23]{staudinger}; siehe auch \cite[(Bearbeiter)§~123 Rn.~12]{muekobgb}.\baselineskip12pt}

Dreizehnte Fußnote. \footnote{Hentschel/\emph{König}/Dauer, § 1 StVG Rn. 1.}

Vierzehnte Fußnote. \footcite[Rn. 3]{viiizr255.17}

Fünfzehnte Fußnote. \footcite[Rn. 15]{1str346.18}

Sechsehnte Fußnote. \footcite[3535]{1str346.18.2}

Siebzehnte Fußnote. \footnote{\cite[56]{laack:infra}; siehe auch \cite[401]{beckemper:unvernunft}.\baselineskip12pt}

\section{Noch ein Abschnitt}
\blindtext

\subsection{Ein Unterabschnitt}
\blindtext

\subsubsection{Ein Unterabschnitt}
\blindtext

\subthreesection{Ein Unterabschnitt}
\blindtext

\subfoursection{Ein Unterabschnitt}
\blindtext

\subfivesection{Ein Unterabschnitt}
\blindtext

\subsixsection{Ein Unterabschnitt}
\blindtext

\subsevensection{Ein Unterabschnitt}
\blindtext

%%\subeightsection{Ein Unterabschnitt}
%%\blindtext

\chapter{Noch ein Kapitel}
Wie in Kapitel~\ref{chap:ErstesKapitel} auf S.~\pageref{chap:ErstesKapitel} \ldots

Wie in Kapitel~\xref{chap:ErstesKapitel} \ldots

%% Bibliographie am Ende auf neuer Seite
%% falls oben gesetzt, dann hier auskommentieren
\newpage
\sloppy
\addcontentsline{toc}{chapter}{Verzeichnis der exemplarisch verwendeten Literatur} % fügt Lit-verz. ins Inhaltsverzeichnis ein
\printbibliography[nottype=jurisdiction]

\end{document}
